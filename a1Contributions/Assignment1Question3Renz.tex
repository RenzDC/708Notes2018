%RenzDelaCruz
\documentclass[12pt, letterpaper, oneside, article]{memoir}
\usepackage[USenglish]{babel}
\usepackage{graphicx}
\usepackage[numbers, square, sort&compress]{natbib}
\usepackage{setspace}
\usepackage{verbatim}
\usepackage{eso-pic}
\usepackage{makeidx}
\citeindextrue
\makeindex
\usepackage[T1]{fontenc}
\renewcommand*\rmdefault{phv}
\usepackage[sumlimits, intlimits, namelimits]{amsmath}
\usepackage[italic]{mathastext}
\usepackage[sfdefault=fav,scaled=0.875]{isomath}
\usepackage{chngcntr}
\counterwithout{equation}{chapter}
\chapterstyle{article}
\setsecnumdepth{subsubsection}
\settocdepth{subsubsection}
\usepackage{indentfirst}
\newlength{\myparindent}
\setlength{\myparindent}{\parindent}
\setlength{\textheight}{0.75\paperheight}
\setlength{\textwidth}{0.66\paperwidth}
\setlength{\oddsidemargin}{1.75cm}
\setlength{\evensidemargin}{\oddsidemargin}
\makeevenfoot{plain}{}{}{\thepage}
\makeoddfoot{plain}{}{}{\thepage}

\begin{document}

Assignment
\vspace{5mm}

Question 3. Prove that for a system with Hamiltonian H, the value of the Hamiltonian (the internal energy of the system) remains constant over time, along solutions of the equations of motion generated by H. 
\vspace{5mm}

\chapter{Conservation}

Represent L (Lagrangian) and H (Hamiltonian) as:
$$\dot p_i=-\frac{\partial H}{\partial q_i}\equiv\frac{\partial L}{\partial q_i}=0\rightarrow p_i=const.,$$
which leads to constant motion.
\vspace{5mm}
Along a path of motion
$$\frac{dH}{dt}=\frac{\partial H}{\partial q_i}\dot{q_i}+\frac{\partial H}{\partial p_i}\dot{p_i}+\frac{\partial H}{\partial t}$$
$$=\frac{\partial H}{\partial q_i}\Big(\frac{\partial H}{\partial p_i}\Big)+\frac{\partial H}{\partial p_i}\Big(-\frac{\partial H}{\partial q_i}\Big)+\frac{\partial H}{\partial t}$$
$$=\frac{\partial H}{\partial t}$$
\vspace{5mm}
So if \textit{H} does not explicitly depend on \textit{t}, $\frac{\partial H}{\partial t}=0$ and so $H=const.$
\vspace{15mm}
\chapter{Phase Space}
\vspace{5mm}
A particular motion from Hamilton's equation can give a path or curve in space. A classical state of a system can be defined at time \textit{t} by a point in 2f dimension $(\vec{q},\vec{p})$.
\vspace{5mm}
The phase velocity, $\vec{v}$, is represented as $$\vec{v}\equiv\begin{bmatrix}
\vec{\dot q}\\
\vec{\dot p}
\end{bmatrix}
=
\begin{bmatrix}
\vec{\nabla_p}H\\
-\vec{\nabla_q}H
\end{bmatrix},
$$
If \textit{H} is independent of \textit{t}, then path lines are on constant energy surfaces.

\vspace{20mm}

(Not sure if this is enough,"sighs")\\\\
-Renz Dela Cruz
\end{document}

