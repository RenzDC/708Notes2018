\documentclass[12pt,a4paper]{article}
\usepackage{graphicx}
\usepackage{amsmath,cases}
\usepackage{booktabs}
\usepackage[top=1in,left=0.3in,right=0.5in,bottom=1in]{geometry}
\usepackage{caption}
\usepackage{changepage}
\usepackage{esvect}
\usepackage{array}
\newcolumntype{L}[1]{>{\raggedright\let\newline\\\arraybackslash\hspace{0pt}}m{#1}}
\newcolumntype{C}[1]{>{\centering\let\newline\\\arraybackslash\hspace{0pt}}m{#1}}
\newcolumntype{R}[1]{>{\raggedleft\let\newline\\\arraybackslash\hspace{0pt}}m{#1}}
\captionsetup{skip=0pt}
\captionsetup{labelformat=empty}

\setlength{\parindent}{0em}
\setlength{\parskip}{0em}

\begin{document}
	\title{Assignment 1, Physics 708}
	\author{Stephen Chung}
	\date{}
	\maketitle
	
\begin{enumerate}
	\setcounter{enumi}{2}
	\item For a system with the Hamiltonian H, the Hamiltonian is only conserved if it satisfies the following expression:
	\[\frac{dH}{dt} = \frac{\partial H}{\partial p} \frac{\partial p}{\partial t} + \frac{\partial H}{\partial q}\frac{\partial q}{\partial t} + \frac{\partial H}{\partial t}\]
    Note that the equation of motion gives us:
    \begin{align*}
    &\frac{\partial H}{\partial p} = \dot{q} \\
    &\frac{\partial H}{\partial q} = -\dot{p}
    \end{align*}
    Which gives us the final expression as:
    \[\frac{dH}{dt} = \frac{\partial H}{\partial t}\]
    
    Note that for an idealised gas or for classical motions (such as a pendulum), the Hamiltonian does not explicitly depend on time so our final expression gives:
    
    \[\frac{dH}{dt} = \frac{\partial H}{\partial t}=0\]
    
    We consider Liouvilles theorem for a large collection of particles as we are unable to identify a specific point in phase space corresponding to the state of the system. By considering an ensemble of equivalent systems, we have that each representative point corresponds to a single system of the ensemble, while the motion of a particular point represent the independent motion of the system. These two paths must not cross as it would then violate the classical mechanics of the system i.e. the system will evolve in more than one direction (or else we wouldn't be able to solve Q1).\\
    This ultimately leads to the ergodicity of the system, where the system will have the same behaviour averaged over time as averaged over space within its phase space i.e. all accessible microstates will be equally probable over a long period of time.
    \setcounter{enumi}{4}
    \newpage
    \item \subsection*{Third Law}
    From the second law, we are able to determine the entropy in terms of T by evaluating the integral:
    \[\int_{V_1}^{V_2}\frac{dQ_{\text{rev}}}{T}\bigg|_T = S(V_2,T) - S(V_1,T)\]
    of some system. We repeat this process while constantly lowering the temperature all the way to zero temp such that we get what Nernst postulated: $\lim\limits_{T\rightarrow 0}\Delta S_{V_1\rightarrow V_2}(T) \rightarrow 0$\\
    As $T\rightarrow0$, entropy becomes more and more independent of its coordinates. We can hypothesize that the entropy of all substances at T=0 is the same universal constant set to zero.\\
    We have small evidence for doing this for different substances.\\
    \\
    Consider Allotropic states of sulphur:
    \vspace{-4mm}
    \begin{figure}[!hbt]
    	\begin{center}
    		\includegraphics[scale=0.05]{state}\par
    	\end{center}
    \end{figure}
    \vspace{-5mm}
    
    We have a monoclinic state and a rhombohedral state defined by its heat capacities $C_V^m$ and $C_V^{\rho}$ respectively. By cooling the substance very slowly, we can observe a transition at a temperature $T_c$, releasing latent heat $L$. By cooling the substance very quickly, we can avoid this transition and maintain in metastable equilibrium. If we desire to obtain the entropy at a temperature slightly above $T_c$, we have two possible paths given:
    \begin{align*}
    &1: S(T_c^{+}) = \int_{0}^{T_c^{+}}\frac{dTC_v^{m}(T)}{T} + S_m(0)\\
    &2: S(T_c^{+}) = \int_{0}^{T_c^{-}}\frac{dTC_v^{\rho}(T)}{T} + \frac{L}{T_c} + S_\rho(0)
    \end{align*}
    Which from measurements have verified that $S_m(0) = S_\rho(0)  = 0$ Which tells us that entropy is independent at $T = 0$\\
    \\
    Consider the consequences:
    \begin{align*}
    &(1) \lim\limits_{T\rightarrow 0}S(V,T) = 0  \rightarrow \lim\limits_{T\rightarrow 0}\frac{\partial S}{\partial V}\bigg|_T =0\\
    &(2) \alpha = \frac{1}{V}\frac{\partial V}{\partial T}\bigg|_P = -\frac{1}{V}\frac{\partial S}{\partial P}\bigg|_T \rightarrow 0\\
    &(3) S(T,V) - S(0,V) = \int_{0}^{T}\frac{C_vdT'}{T'}\\
    &(4) \text{Unattainability of $T=0$}
    \end{align*}
    The first consequence represent another expression for entropy leading to zero as the temperature tends to zero.\\
    The second consequence represent the extensivity of a system, manipulated by the maxwell's relation which tends to zero as the temperature tends to zero.\\
    The third consequence represents a finite range of possible entropy values for a given finite temperature value. But if the heat capacity is constant, we will get a logarithmic value of entropy which blows up as the temperature tends to zero. This can only be corrected for $\lim\limits_{T\rightarrow 0} C_v(T) \rightarrow 0$\\
    The fourth consequence implies that it is impossible to cool any system to absolute zero temperature in a finite number of steps. 
\end{enumerate}
	
\end{document}